\begin{UseCase}{CU11}{Consultar medicamento por sucursal}{
		Es necesario hacer una consulta de los medicamentos por sucursal ya que no todas las sucursales tienen las mismas cantidades de medicamentos disponibles y tambien las ventas de cada sucursal varian.
	}
		\UCitem{Versión}{\color{Gray}0.1.4}
		\UCitem{Autor}{\color{Gray}Zapata Jasso José Rodolfo}
		\UCitem{Supervisa}{\color{Gray}Miguel}
		\UCitem{Actor}{Dueño, Supervisor, Cajero}
		\UCitem{Propósito}{Conocer las cantidades de medicamentos que hay en cada sucursal.}
		\UCitem{Entradas}{Lote, ID del Proveedor, Nombre de la Sucursal, ID del Medicamento}
		\UCitem{Origen}{Teclado}
		\UCitem{Salidas}{Lote, ID del Proveedor, Nombre de la Sucursal, ID del Medicamento, Unidades a Recibir del Medicamento, Nombre de la sucursal, Dirección de la sucursal, Teléfono de la sucursal, Estado de la sucursal.}
		\UCitem{Destino}{Pantalla}
		\UCitem{Precondiciones}{Deben existir sucursales en el sistema.}
		\UCitem{Postcondiciones}{Conocer el estado de los medicamentos de cada sucursal.}
		\UCitem{Errores}{La pagina sea inaccesible por el momento debido a fallas con los servidores.}
		\UCitem{Observaciones}{}
		\UCitem{Estado}{Revisión}
		\UCitem{Viene de}{CU0}
	\end{UseCase}
%--------------------------------------
	\begin{UCtrayectoria}{Principal}
		\UCpaso Incluye el caso de uso \UCref{CU0} paso 11
		\UCpaso[\UCactor] Selecciona La opción Medicamentos en la \IUref{01}{Pantalla Principal} presionando el botón \IUbutton{Medicamentos}.
		\UCpaso Genera y Despliega las opciones de Medicamntos \IUref{01}
		\UCpaso [\UCactor] Una ves ya en la \IUref{17}{Tabla Medicamentos} se selecciona la flecha junto al texto "Nombre" para cambiar la opción a Sucursal  presionando el botón \IUbutton{Sucursal}, despues se selecciona el apartado  de "Texto a buscar" y se ingresa el nombre de la sucursal que desea buscar y da clic en el botón \IUbutton{Buscar}. \Trayref{A}
		\UCpaso Despliega solo las sucursales con el nombre ingresado en la \IUref{17}{Proveedor Formulario} con sus debidos datos.
	\end{UCtrayectoria}


%-------------------------------------------------------------------------
\begin{UCtrayectoriaA}{A}{Empleado no encontrado.}
			\UCpaso Muestra el Mensaje {\bf MSG1-}`` [{\em Error en la operación}]Verifique que los datos ingresados correspondan con los datos solicitados en la  \IUref{IU17}{Tabla Medicamentos}.''.
			\UCpaso Continúa en el paso 4 del \UCref{CU11}.
		\end{UCtrayectoriaA}
%-------------------------------------------------------------------------
