\begin{UseCase}{CU13}{Desactivar el estado del Medicamento}{
		Se requiere desactivar un medicamento por diversos motivos.
	}
		\UCitem{Versión}{\color{Gray}0.1}
		\UCitem{Autor}{\color{Gray}Vázquez Cruz Fernando Darwin }
		\UCitem{Supervisa}{\color{Gray}}
		\UCitem{Actor}{\hyperlink{Alumno}{Dueño}}
		\UCitem{Propósito}{Que el Dueño pueda desactivar un medicamento que pueda causar problemas en las sucursales.}
		\UCitem{Entradas}{Nombre del medicamento, código de barras del medicamento o ingrediente activo del medicamento.}
		\UCitem{Origen}{Teclado, Mouse}
		\UCitem{Salidas}{Mensaje de confirmación de desactivación.}
		\UCitem{Destino}{Pantalla}
		\UCitem{Precondiciones}{El medicamento debe de existir.}
		\UCitem{Postcondiciones}{Se desactivará el medicamento.}
		\UCitem{Errores}{El medicamento no puede ser desactivado.}
		\UCitem{Tipo}{Caso de uso primario.}
		\UCitem{Observaciones}{}
		\UCitem{Estado}{Revisión}
	\end{UseCase}
%--------------------------------------
\begin{UCtrayectoria}{Principal}
		\UCpaso Se incluye el caso de uso \UCref{CU12}.
		\UCpaso[\UCactor] Selecciona el tipo de búsqueda que hará.
		\UCpaso[\UCactor] Da clic en la barra de búsqueda.
		\UCpaso [\UCactor] Ingresa nombre del medicamento, código de barras o ingrediente activo según corresponda. 
		\UCpaso[\UCactor] Da clic en el \IUbutton {+ Buscar }.
		\UCpaso Despliega una lista que coincida con la búsqueda realizada. \Trayref{A}
		\UCpaso[\UCactor] Selecciona el medicamento de la lista.\Trayref{B}
		\UCpaso[\UCactor] Da clic en la \IUref {IU9} {Botton Desactivar}.
		\UCpaso Despliega el mensaje \bf {+ MSG2}.
		\UCpaso[\UCactor] El Dueño da clic en el \IUbutton {+ Si, Desactivalo!} para confirmar la operación.
		\UCpaso Actualiza el estado del medicamento y la lista de medicamentos.
		\UCpaso Re-direcciona al \UCactor a la \IUref {IU17} {Tabla de Medicamentos}.
	
	\end{UCtrayectoria}


		\begin{UCtrayectoriaA}{A}{El Medicamento no se encuentra en la base de datos}
			\UCpaso[\UCactor] El Dueño busca el medicamento a seleccionar y no lo encuentra.
			\UCpaso[\UCactor] El Dueño regresa a la \IUref {IU17} {Tabla de Medicamentos}.
			\UCpaso Continua en el paso 2 del \UCref{CU13}.
		\end{UCtrayectoriaA}
		

%--------------------------------------