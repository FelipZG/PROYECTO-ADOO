\begin{UseCase}{CU36}{Registrar datos de un cliente}{
		El cajero registra datos de los clientes en el sistema.
	}
		\UCitem{Versión}{\color{Gray}0.1}
		\UCitem{Autor}{\color{Gray}Felipe Zamora Gachuz}
		\UCitem{Supervisa}{\color{Gray}.}
		\UCitem{Actor}{Cajero}
		\UCitem{Propósito}{Ingresar al Sistema para poder registra un nuevo cliente preferente}%no me acuerod como se llaman los clientes registrados
		\UCitem{Entradas}{Correo Electrónico del cliente numero telefonico}%nose como  esten viendo este caso revisar
		\UCitem{Origen}{Teclado}
		\UCitem{Salidas}{Se pueden observar los datos ingresados dentro de \IUref{IUregistro de cliente}{Pantalla del Cajero}}
		\UCitem{Destino}{Registro en el sistema BD}
		\UCitem{Precondiciones}{No se debe repetir el correo electronico, estar en las pantallas del cajero}
		\UCitem{Postcondiciones}{El cliente tendra un registro unico en el sistema}
		\UCitem{Errores}{Error de conexion, el correo electronico ya este registrado}
		\UCitem{Observaciones}{}
		\UCitem{Estado}{En revision}
	\end{UseCase}
%--------------------------------------
	\begin{UCtrayectoria}{Principal}
		\UCpaso[\UCactor] Tiene que cumplir el caso de uso \IUref{IU1}{Login} en el apartado del cajero.
		\UCpaso[\UCactor] Tiene que estar en la pestaña de nuevo cliente.
		\UCpaso El sistema despliega un formulario de registro de nuevos clientes\IUref{IU6}{Formulario Clientes}. 
		\UCpaso [\UCactor] proporciona los datos requeridos en el formulario.
		\UCpaso Verifica que el correo proporcionado cumpla con el formato ejemplo@ejemplo.com \Trayref{A}.
		\UCpaso Presiona el boton Aceptar.
		\UCpaso Verifica que el correo proporcionado no este ya registrado. \Trayref{B}.
		\UCpaso El sistema despriega la pantalla  {\bf MSG0-``Operación Exitosa''}.
		\UCpaso Despliega la pantalla principal del actor.	
	\end{UCtrayectoria}

%--------------------------------------		
	\begin{UCtrayectoriaA}{A}{El Correo no esta Correcto}
			\UCpaso Muestra el Mensaje {\bf MSG03-``\em Correo con formato incorrecto Introduzca un correo con el formato ejemplo@ejemplo.com .''.}
			\UCpaso Continúa en el paso 3 del \UCref{CU36}.
		\end{UCtrayectoriaA}
%----------------------------------------
		\begin{UCtrayectoriaA}{B}{El Correo electronico ya esta registrado}
			\UCpaso Muestra la pantalla {\bf MSG1-``Error en la Operación''}.
			\UCpaso[\UCactor] Oprime el botón \IUbutton{Aceptar}.
			\UCpaso[] Continua en el paso 4 del \UCref{CU36}.
		\end{UCtrayectoriaA}		
%--------------------------------------
		