\begin{UseCase}{CU18}{Activar estado de sucursal}{
		Se activa cuando la sucursal esta en operación.
	}
		\UCitem{Versión}{\color{Gray}0.1.2}
		\UCitem{Autor}{\color{Gray}Zapata Jasso José Rodolfo}
		\UCitem{Supervisa}{\color{Gray}Correa Medina Carlos Miguel}
		\UCitem{Actor}{\hyperlink{Alumno}{Dueño}}
		\UCitem{Propósito}{Disponibilidad de administrar la sucursal cuando esta en operación.}
		\UCitem{Entradas}{Nombre del Empleado, Id de Empleado}
		\UCitem{Origen}{Teclado}
		\UCitem{Salidas}{No Aplica.}
		\UCitem{Destino}{Pantalla}
		\UCitem{Precondiciones}{Debe existir dicha sucursal y debe estar desactivada.}
		\UCitem{Postcondiciones}{El estado de la sucursal deberá cambiar de desactivada a activada.}
		\UCitem{Errores}{La pagina sera inaccesible por el momento debido a fallas con los servidores.}
		\UCitem{Tipo}{Caso de uso primario}
		\UCitem{Observaciones}{}
		\UCitem{Estado}{Aprobado}
	\end{UseCase}


%--------------------------------------


	\begin{UCtrayectoria}{Principal}
		\UCpaso Se incluye el caso de uso \UCref{CU1}.
		\UCpaso Se incluye el caso de uso \UCref{CU22}
		\UCpaso[\UCactor] Selecciona la  Activar sucursal, de la sucursal que desea activar presionando el \IUref{IU10}{boton activar} que esta representado con una palomita azul.		
		\UCpaso Genera y despliega una pantalla de confirmación.  {\bf MSG3-}.
		\UCpaso[\UCactor] Confirma la acción seleccionando el botón\IUbutton{Si, Activar} . \Trayref{A}
		\UCpaso Guarda los cambios y redirige al [\UCactor] a la  \IUref{01}{Pantalla Principal de Dueño}.
	\end{UCtrayectoria}


%-------------------------------------------------------------------------


\begin{UCtrayectoriaA}{A}{Error al cambiar el estado.}
			\UCpaso Muestra el Mensaje {\bf MSG1-}`` [{\em error en la operación}] ''.
			
		\end{UCtrayectoriaA}
%-------------------------------------------------------------------------

