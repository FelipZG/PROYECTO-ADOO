\begin{UseCase}{CU15}{Modificar paquete de descuento}{
		Realizar cambios en los paquetes de descuento.
	}
		\UCitem{Versión}{\color{Gray}0.1.2}
		\UCitem{Autor}{\color{Gray}Zapata Jasso José Rodolfo}
		\UCitem{Supervisa}{\color{Gray}Correa Medina Carlos Miguel}
		\UCitem{Actor}{\hyperlink{Alumno}{Dueño}}
		\UCitem{Propósito}{Realizar cambios en los paquetes de descuentos ya publicados.}
		\UCitem{Entradas}{Nombre del Empleado, Id de Empleado}
		\UCitem{Origen}{Teclado}
		\UCitem{Salidas}{No Aplica.}
		\UCitem{Destino}{Pantalla}
		\UCitem{Precondiciones}{Debe existir un paquete de descuento registrado previamente.}
		\UCitem{Postcondiciones}{El paquete de descuento deberá cambiar a como estaba previamente.}
		\UCitem{Errores}{La pagina sera inaccesible por el momento debido a fallas con los servidores, problemas al registrar cambios ne los paquetes.}
		\UCitem{Tipo}{Caso de uso primario}
		\UCitem{Observaciones}{}
		\UCitem{Estado}{Aprobado}
	\end{UseCase}


%--------------------------------------


	\begin{UCtrayectoria}{Principal}
		\UCpaso Se incluye el caso de uso \UCref{CU1}.
		\UCpaso Se incluye el caso de uso \UCref{Cu42} 
		\UCpaso[\UCactor] Selecciona la opción editar paquetes de descuento presionando el botón \IUref{IU11}{boton editar} que esta representado con un lapiz amarillo.
		\UCpaso Genera y Despliega el paquete de descuentos para su edición. 
		\UCpaso[\UCactor] Realiza los cambios que desea modificar en el paquete de descuentos y confirma al presiona el botón \IUbutton{Actualizar} . \Trayref{A}
		\UCpaso Redirige al [\UCactor] a la  \IUref{01}{Pantalla Principal de Dueno}.
	\end{UCtrayectoria}


%-------------------------------------------------------------------------


\begin{UCtrayectoriaA}{C}{Algun campo tiene un error.}
			\UCpaso Muestra el Mensaje {\bf MSG1-}`` [{\em Hubo un error en la operación }] revisa que los campos sean llenados correctamente.''.
			
		\end{UCtrayectoriaA}
