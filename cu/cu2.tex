\begin{UseCase}{CU2}{Recuperar Contraseña}{
		El empleado puede cometer la equivocación de perder la contraseña asociada a la cuenta con la que ingresa al sistema, por lo cual se requiere de una forma para poder recuperar la contraseña. La recuperación de realizara mediante un correo electrónico el cual se le enviara al empleado con la posibilidad de reasignar una nueva contraseña
	}
		\UCitem{Versión}{\color{Gray}0.1}
		\UCitem{Autor}{\color{Gray}Enrique Aguilera Rosas Landa}
		\UCitem{Supervisa}{\color{Gray}Correa Medina Carlos Miguel}
		\UCitem{Actor}{\hyperlink{Empleado}{Supervisor, Cajero}}
		\UCitem{Propósito}{Conceder el acceso al sistema de nuevo mediante la recuperación de la contraseña.}
		\UCitem{Entradas}{Correo Electrónico.}
		\UCitem{Origen}{Teclado}
		\UCitem{Salidas}{Contraseña Asociada con la cuenta.}
		\UCitem{Destino}{Pantalla}
		\UCitem{Precondiciones}{El empleado debe estar registrado en el sistema y que no recuerde su contraseña asociada .}
		\UCitem{Postcondiciones}{El empleado recuperara su contraseña y su acceso al sistema.}
		\UCitem{Errores}{Que el acceso a la pagina sea incorrecto debido a razones de Errores De conexión o Mantenimiento de los Servidores, Que el usuario no este registrado en el sistema}

		\UCitem{Observaciones}{}
		\UCitem{Estado}{Corrección}
	\end{UseCase}
%--------------------------------------
	\begin{UCtrayectoria}{Principal}
		\UCpaso Incluye el caso de uso \UCref{CU1}.
		\UCpaso[\UCactor] Presiona el botón \IUbutton{¿Olvidaste tu Password?}.
		\UCpaso [\UCactor] Proporciona su Correo Electrónico asociado a la contraseña perdida y  Confirma la operación presionando el botón  \IUbutton{Aceptar}.		
		\UCpaso Verifica que el correo proporcionado cumpla con el formato ``Ejemplo@ejemplo.com'' \Trayref{A}
		\UCpaso Busca la cuenta asociada al correo ingresado. \Trayref{B}
		\UCpaso Verifica que dicha cuenta este activa. \Trayref{C}.
		\UCpaso Envía un correo electrónico al correo proporcionado; el cual contara con un link que lleva a la \IUref{12}{Recuperar Contraseña} para asignar una nueva contraseña. \Trayref{D}.
		\UCpaso Redirecciona al \UCactor a la  \IUref{IU1}{Pantalla Principal}.
	\end{UCtrayectoria}

%--------------------------------------		
	\begin{UCtrayectoriaA}{A}{El Correo no esta Correcto}
			\UCpaso Muestra el Mensaje {\bf MSG01-}``Error en la Operación [{\em correo con formato incorrecto}] Introduzca un correo con el formato xxx@xx.xx .''.
			\UCpaso Continúa en el paso 3 del \UCref{CU2}.
		\end{UCtrayectoriaA}
%----------------------------------------
		\begin{UCtrayectoriaA}{B}{El \UCactor no esta registrado}
			\UCpaso Muestra el Mensaje {\bf MSG01-}``Error en la operación [{\em Usuario no Encontrado}] El usuario y/o contraseña no existen .''.
			\UCpaso[\UCactor] Oprime el botón \IUbutton{Aceptar}.
			\UCpaso[] Continua en el paso 3 del \UCref{CU2}.
		\end{UCtrayectoriaA}		
%--------------------------------------
		\begin{UCtrayectoriaA}{C}{La cuenta a la que intenta acceder no esta activa}
			\UCpaso Muestra el Mensaje {\bf MSG01-}``Error en la operación [{\em Cuenta Desactivada}] Contacta con el Dueño para resolver el problema .''.
			\UCpaso[\UCactor] Oprime el botón \IUbutton{Aceptar}
			\UCpaso Continua en el paso 3 del \UCref{CU2}.
		\end{UCtrayectoriaA}
%--------------------------------------
	\begin{UCtrayectoriaA}{D}{Correo Electrónico de recuperación de contraseña no se envió}
			\UCpaso Muestra el Mensaje {\bf MSG01-}``Error en la operación [{\em Correo no enviado}] Revisa tu conexión y vuelve a enviar el mensaje .''.
			\UCpaso[\UCactor] Oprime el botón \IUbutton{Aceptar}
			\UCpaso Continua en el paso 7 del \UCref{CU2}.
		\end{UCtrayectoriaA}
%--------------------------------------


		
