\begin{UseCase}{CU3}{Modificar datos de  Empleado}{
		Los datos del empleado se tienen que modificar, por inconsistencia en los datos registrados en el sistema con sus datos actuales
	}
		\UCitem{Versión}{\color{Gray}0.1.3}
		\UCitem{Autor}{\color{Gray}Aguilera Rosas Landa Enrique}
		\UCitem{Supervisa}{\color{Gray}Correa Medina Carlos Miguel}
		\UCitem{Actor}{\hyperlink{Alumno}{Dueño}}
		\UCitem{Propósito}{Evitar problemas con los empleados por datos erróneos guardados en el sistema.}
		\UCitem{Entradas}{Nombre completo del solicitante, Edad, Dirección, Curp, RFC, Puesto , Sucursal}
		\UCitem{Origen}{Teclado}
		\UCitem{Salidas}{No Aplica.}
		\UCitem{Destino}{Pantalla}
		\UCitem{Precondiciones}{El empleado debe de estar dado de alta en el sistema.}
		\UCitem{Postcondiciones}{Los datos del empleado serán diferentes.}
		\UCitem{Errores}{Exista algún duplicado en los datos del empleado, sus cambios no sean guardados}
		\UCitem{Observaciones}{}
		\UCitem{Estado}{Corrección}
	\end{UseCase}
%--------------------------------------
	\begin{UCtrayectoria}{Principal}
		\UCpaso Incluye el  caso de uso \UCref{CU1}.
		\UCpaso[\UCactor] Selecciona La opción de ver Lista de  Empleados presionando el botón \IUbutton{Ver Lista de Empleado}.
		\UCpaso Incluye el caso de uso \UCref{CU5}.
		\UCpaso[\UCactor] Introduce el Nombre del empleado a buscar en el campo de Búsqueda y presiona el botón \IUbutton{Buscar} \Trayref{A} .
		\UCpaso Genera y Despliega una lista que coincida con la búsqueda realizada
		\UCpaso[\UCactor] Selecciona la opción editar datos Del empleado Deseado presionando\IUbutton{Editar}.
		\UCpaso Genera el formulario Datos del empleado con todos los datos del empleado y los despliega.
		\UCpaso[\UCactor] Cambia los datos que el empleado necesita modificar y guarda los cambios presionando el botón \IUbutton{Actualizar} \Trayref{B}.
		\UCpaso Redirige al [\UCactor] a la  \IUref{01}{Pantalla Principal}.
	\end{UCtrayectoria}


%-------------------------------------------------------------------------


\begin{UCtrayectoriaA}{A}{Empleado no encontrado.}
			\UCpaso Muestra el Mensaje {\bf MSG01-}``Error en la Operación [{\em Empleado no encontrado}] revisa que los campos sean llenados correctamente.''.
			\UCpaso Continúa en el paso  del \UCref{CU3}.
		\end{UCtrayectoriaA}
%-------------------------------------------------------------------------


\begin{UCtrayectoriaA}{B}{Algún campo del Empleado tiene un error.}
			\UCpaso Muestra el Mensaje {\bf MSG01-}``Error en la Operación [{\em Error en un dato del Empleado}] revisa que los campos sean llenados correctamente.''.
			\UCpaso Continúa en el paso 9 del \UCref{CU3}.
		\end{UCtrayectoriaA}
