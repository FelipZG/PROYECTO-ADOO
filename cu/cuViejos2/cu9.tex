\begin{UseCase}{CU9}{Asignar Sucursal(es) a Supervisor.}{
		Se requiere un método sencillo y rápido para asignar diferentes sucursales a un supervisor.
	}
		\UCitem{Versión}{\color{Gray}0.1}
		\UCitem{Autor}{\color{Gray}Vázquez Cruz Fernando Darwin }
		\UCitem{Supervisa}{\color{Gray}}
		\UCitem{Actor}{\hyperlink{Alumno}{Dueño}}
		\UCitem{Propósito}{Que el Dueño pueda asignarles a sus nuevos supervisores una o más sucursales para tener un mejor control de las franquicias.}
		\UCitem{Entradas}{Datos del supervisor, nombre de las sucursales.}
		\UCitem{Origen}{Teclado, Mouse}
		\UCitem{Salidas}{Mensaje de confirmación de asignación.}
		\UCitem{Destino}{Pantalla}
		\UCitem{Precondiciones}{El dueño debe dar clic en el \IUbutton {+ Supervisor } en el paso 5 del caso de uso \UCref{CU8}.}
		\UCitem{Postcondiciones}{Se asignará un nuevo supervisor a una o más sucursales.}
		\UCitem{Errores}{La sucursal no puede ser asignada.}
		\UCitem{Tipo}{Caso de uso primario.}
		\UCitem{Observaciones}{}
		\UCitem{Estado}{Revisión}
	\end{UseCase}
%--------------------------------------
	\begin{UCtrayectoria}{Principal}
		\UCpaso Se extiende del caso de uso \UCref{CU8} paso 5.
		\UCpaso Despliega la \IUref {IU13} {Tabla de Supervisores}.
		\UCpaso[\UCactor] Presiona el \IUbutton {+ Nuevo }.
		\UCpaso Despliega la \IUref {IU2} {Formulario Supervisor}.
		\UCpaso [\UCactor] Llena los campos requeridos para registrar un nuevo supervisor. 
		\UCpaso [\UCactor] Da clic en las sucursales en las que se desea asignar al supervisor.
		\UCpaso [\UCactor] Da clic en el botón.\IUbutton{Guardar}.
		\UCpaso Despliega el mensaje \bf {+ MSG0}. \Trayref{A}
		\UCpaso Guarda los datos del nuevo empleado y actualiza la lista de empleados.
		\UCpaso[\UCactor] Da clic en el \IUbutton {+ Ok }.
		\UCpaso Re-direcciona al \UCactor a la \IUref {IU13} {Tabla de Supervisores}.
	
	\end{UCtrayectoria}


		\begin{UCtrayectoriaA}{A}{El supervisor no puede ser asignado a una sucursal}
			\UCpaso Despliega el mensaje \bf {+ MSG1 }.
			\UCpaso[\UCactor] Da clic en el \IUbutton {+ Ok }.
			\UCpaso Continua en el paso 3 del \UCref{CU9}.
		\end{UCtrayectoriaA}

%--------------------------------------