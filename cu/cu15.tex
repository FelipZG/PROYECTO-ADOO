\begin{UseCase}{CU15}{Agregar nueva sucursal}{
		Se requiere una forma para agregar una nueva sucursal en el sistema.
	}
		\UCitem{Versión}{\color{Gray}0.1.3}
		\UCitem{Autor}{\color{Gray}Aguilera Rosas Landa Enrique}
		\UCitem{Supervisa}{\color{Gray} Correa Medina Carlos Miguel}
		\UCitem{Actor}{\hyperlink{Alumno}{Dueño}}
		\UCitem{Propósito}{Tener el control de una nueva sucursal y que este disponible para operar.}
		\UCitem{Entradas}{Nombre de la sucursal, Dirección de la sucursal,Teléfono de la sucursal}
		\UCitem{Origen}{Teclado}
		\UCitem{Salidas}{Nombre de la sucursal, Dirección de la sucursal,Teléfono de la sucursal, Estado de la sucursal.}
		\UCitem{Destino}{Pantalla}
		\UCitem{Precondiciones}{Debe haber una sucursal nueva física, que aun no este registrada en el sistema.}
		\UCitem{Postcondiciones}{Aparecerá una nueva sucursal en el sistema.}
		\UCitem{Errores}{La pagina sea inaccesible por el momento debido a fallas con los servidores.}
		\UCitem{Observaciones}{.}
		\UCitem{Estado}{En revisión}
		\UCitem{Viene de}{CU0}
	\end{UseCase}


%--------------------------------------


	\begin{UCtrayectoria}{Principal}
		\UCpaso Se incluye el caso de uso \UCref{CU0}
		\UCpaso[\UCactor] Selecciona la opción Sucursales en la  \IUref{01}{Pantalla Principal}
		\UCpaso Genera y Despliega la \IUref{14}{Tabla Sucursales}
		\UCpaso[\UCactor] Selecciona la opción para registrar una nueva sucursal presionando el botón \IUbutton{+Nuevo} en la \IUref{14}{Tabla Sucursales}.
		\UCpaso Genera y despliega la \IUref{IU3}{Formulario Sucursal} con los campos vacíos y listos para llenar. 
		\UCpaso[\UCactor] Llena los datos del formulario \IUref{IU3}{Formulario Sucursal}, de manera que cada dato corresponda con la información que se pide y selecciona el botón \IUbutton{Guardar} .   
		\UCpaso Verifica que todos los datos sean llenados correctamente como se solicita en la \IUref{3}{Formulario Sucursal}\Trayref{A}
		\UCpaso Verifica que los datos ingresados no se repitan con los de otras sucursales existentes en el sistema y que la dirección sea una dirección valida buscando en google maps. \Trayref{B} \Trayref{C}
		\UCpaso Genera y despliega la ventana \IUref{MSG0}{Operación Realizada Con Éxito} 
		\UCpaso [\UCactor] Cierra la ventana presionando el \IUbutton{OK}
		\UCpaso Redirige al [\UCactor] a la  \IUref{01}{Pantalla Principal}.
	\end{UCtrayectoria}


%-------------------------------------------------------------------------


\begin{UCtrayectoriaA}{A}{Problemas al guardar datos de la sucursal.}
			\UCpaso Muestra el Mensaje {\bf MSG1-}`` [{\em Error en la operación}]Verifique que los datos ingresados correspondan con los datos solicitados en la  \IUref{IU3}{Formulario Sucursal}.''.
			\UCpaso Continúa en el paso 6 del \UCref{CU15}.
		\end{UCtrayectoriaA}
%-------------------------------------------------------------------------
\begin{UCtrayectoriaA}{B}{Dato Duplicado.}
			\UCpaso Muestra el Mensaje {\bf MSG1-}`` [{\em Error en la operación}] Uno o mas datos ingresados ya existen en el sistema  \IUref{IU3}{Formulario Sucursal}.''.
			\UCpaso Continúa en el paso 6 del \UCref{CU15}.
		\end{UCtrayectoriaA}
%-------------------------------------------------------------------------
\begin{UCtrayectoriaA}{B}{Dirección Invalida.}
			\UCpaso Muestra el Mensaje {\bf MSG1-}`` [{\em Error en la operación}] La dirección ingresada no es una dirección valida  \IUref{IU3}{Formulario Sucursal}.''.
			\UCpaso Continúa en el paso 6 del \UCref{CU15}.
		\end{UCtrayectoriaA}