\begin{UseCase}{CU10}{Cerrar Caja}{
		Conocer si existe un descuadre en las ventas del día, ya sea una diferencia, un faltante o un sobrante.
	}
		\UCitem{Versión}{\color{Gray}0.1.4}
		\UCitem{Autor}{\color{Gray}Zapata Jasso José Rodolfo}
		\UCitem{Supervisa}{\color{Gray}Miguel}
		\UCitem{Actor}{Supervisor}
		\UCitem{Propósito}{Conocer si hay un descuadre al finalizar el día.}
		\UCitem{Entradas}{ID del Proveedor, ID del Cajero}
		\UCitem{Origen}{Teclado}
		\UCitem{Salidas}{Fecha y Hora, Dinero en caja.}
		\UCitem{Destino}{Pantalla}
		\UCitem{Precondiciones}{La sucursal debió laborar dicho día, la caja debió haber sido abierta.}
		\UCitem{Postcondiciones}{Se cerrara la caja.}
		\UCitem{Errores}{La pagina sea inaccesible por el momento debido a fallas con los servidores.}
		\UCitem{Observaciones}{}
		\UCitem{Estado}{Revisión}
		\UCitem{Viene de}{CU0}
	\end{UseCase}
%--------------------------------------
	\begin{UCtrayectoria}{Principal}
		\UCpaso Incluye el caso de uso \UCref{CU0} paso 11
		\UCpaso[\UCactor] Selecciona La opción Cerrar Caja en la \IUref{01}{Pantalla Principal} presionando el botón \IUbutton{Cerrar Caja}.
		\UCpaso Genera y Despliega la tabla  \IUref{IU22}{Cerrar Caja} para ingresar los datos ID Proveedor y el ID del Cajero que termina turno. 
		\UCpaso [\UCactor] Ingresa los datos solicitados y presiona el botón \IUbutton{Aceptar} \Trayref{A}
		\UCpaso Despliega el mensaje {\bf MSG06-} {Confirmar Operación}. 
		\UCpaso [\UCactor] Presiona el botón {Si!}.
		\UCpaso Despliega el mensaje {\bf MSG0-} {Operación Exitosa}
		\UCpaso [\UCactor] Presiona el botón {OK}.
		\UCpaso Despliega la pantalla principal \IUref{01}{Pantalla Principal}.
	\end{UCtrayectoria}


%-------------------------------------------------------------------------
\begin{UCtrayectoriaA}{A}{Error}
			\UCpaso Muestra el Mensaje {\bf MSG1-}`` [{\em Error en la operación}]Verifique que los datos ingresados correspondan con los datos solicitados en la tabla \IUref{IU22}{Cerrar Caja}.''.
			\UCpaso Continúa en el paso 4 del \UCref{CU10}.
		\end{UCtrayectoriaA}
%-------------------------------------------------------------------------
