\begin{UseCase}{CU7}{Activar estado de Empleado}{
		Por razones laborales, a algunos empleados se les reactivara una cuenta suspendida para que puedan trabajar en el sistema de nuevo
	}
		\UCitem{Versión}{\color{Gray}0.1.3}
		\UCitem{Autor}{\color{Gray}Aguilera Rosas Landa Enrique}
		\UCitem{Supervisa}{\color{Gray}Correa Medina Carlos Miguel}
		\UCitem{Actor}{\hyperlink{Alumno}{Dueño}}
		\UCitem{Propósito}{Que el empleado tenga acceso al sistema.}
		\UCitem{Entradas}{Nombre del Empleado, Id de Empleado}
		\UCitem{Origen}{Teclado}
		\UCitem{Salidas}{No Aplica.}
		\UCitem{Destino}{Pantalla}
		\UCitem{Precondiciones}{El empleado debe de estar registrado en el sistema y con un estado desactivado.}
		\UCitem{Postcondiciones}{El empleado ganará su acceso al sistema y su estado cambiara a activado .}
		\UCitem{Errores}{La pagina sea inaccesible por el momento debido a fallas con los servidores, Que el empleado tenga su cuenta activada}
		\UCitem{Observaciones}{}
		\UCitem{Estado}{En revision}
	\end{UseCase}
%--------------------------------------
	\begin{UCtrayectoria}{Principal}
		\UCpaso Incluye el caso de uso \UCref{CU5}.
		\UCpaso[\UCactor] Introduce el Nombre del empleado a buscar en el campo de Búsqueda y presiona el botón \IUbutton{Buscar} .
		\UCpaso Genera y Despliega una lista que coincida con la búsqueda realizada.\Trayref{A}
		\UCpaso[\UCactor] Presiona el botón\IUbutton{Activar Cuenta} del empleado seleccionado.
		\UCpaso Genera y despliega la ventana emergente \IUref{ConfirmaciónActivar} para confirmar la operación de Activar la cuenta seleccionada.
		\UCpaso [\UCactor] Confirma la operación presionando el \IUbutton{Si, Activar}
		\UCpaso Redirige al [\UCactor] a la  \IUref{01}{Pantalla Principal}.
	\end{UCtrayectoria}


%-------------------------------------------------------------------------
\begin{UCtrayectoriaA}{A}{Empleado no encontrado.}
			\UCpaso Muestra el Mensaje {\bf MSG01-}``Error en la Operacion [{\em Empleado no encontrado}] revisa que los campos sean llenados correctamente.''.
			\UCpaso Continúa en el paso 4 del \UCref{CU4}.
		\end{UCtrayectoriaA}
%-------------------------------------------------------------------------
