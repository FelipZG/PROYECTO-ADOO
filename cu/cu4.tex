 \begin{UseCase}{CU4}{Registrar Supervisor}{
		Con el crecimiento de la Franquicia, es necesario contratar nuevo personal, personal encargado de administrar diferentes sucursales, este empleado es el supervisor,el mismo tiene responsabilidades en el sistema, por lo tanto el dueño  necesita de un método sencillo y rápido con el que le pueda incorporar al supervisor al trabajo una vez contratado.
	}
		\UCitem{Versión}{\color{Gray}0.1}
		\UCitem{Autor}{\color{Gray}Correa Medina Carlos Miguel}
		\UCitem{Supervisa}{\color{Gray}.Darwin}
		\UCitem{Actor}{Dueño}
		\UCitem{Propósito}{facilitar la incorporación de un nuevo supervisor, asignado áreas de trabajo de forma sencilla y eficaz.}
		\UCitem{Entradas}{Todos los datos proporcionados en el Formulario \IUref{IU2}{Formulario Supervisor}}
		\UCitem{Origen}{Teclado}
		\UCitem{Salidas}{nueva fila en  \IUref{IU13}{Tabla Supervisores}}
		\UCitem{Destino}{Pantalla}
		\UCitem{Precondiciones}{El Supervisor no debe estar registrado en el sistema,debe existir por lo menos una sucursal registrada en el sistema.}
		\UCitem{Postcondiciones}{la capacidad de supervisores por Sucursal disminuye en tres Sucursales máximo y solo una sucursal como mínimo.}
		\UCitem{Errores}{Que no se cuente con conexión a Internet,no haya energía eléctrica para utilizar un computador,Que no se realice una conexión a la base de datos,que el servidor se caiga,que los datos proporcionados estén erróneos, que no haya cupo en ninguna sucursal.}
		\UCitem{Viene de...}{\UCref{CU0}{Control de acceso}}
		\UCitem{Observaciones}{}
		\UCitem{Estado}{En revisión}
	\end{UseCase}
%--------------------------------------
	\begin{UCtrayectoria}{Principal}
		\UCpaso incluye al caso de uso \UCref{IU0}{Control de Acceso}.
		\UCpaso [\UCactor] presiona el botón \IUbutton{Empleados}
		\UCpaso verifica que los permisos de usuario sean permisos de Dueño. \Trayref{A}
		\UCpaso Despliega una lista con los botones \IUbutton{Supervisor} y 
		\IUbutton{Cajero}
		\UCpaso [\UCactor] presiona el botón \IUbutton{Supervisor}
		\UCpaso Muestra la pantalla \IUref{IU13}{Tabla supervisores}
		\UCpaso [\UCactor] Presiona el botón \IUbutton{+Nuevo}
		\UCpaso Genera una lista con los nombres de las Sucursales que hay están en el sistema y la muestra en \IUref{IU2}{Formulario Supervisor} como una ``checklist"
		\UCpaso Marca las ``checklist" ,generadas en el paso anterior, de las sucursales que ya no tienen cupo para más Supervisores y las des-habilita para no reasignar Sucursales.\Trayref{B} 	 	 
		\UCpaso Muestra la pantalla \IUref{IU2}{Formulario Supervisor}
		\UCpaso [\UCactor] llena los campos: nombre,Apellido Paterno,Apellido Materno, Email, Teléfono y Salario.
		\UCpaso [\UCactor] De las Sucursales desmarcadas selecciona las sucursales de las que se hará cargo el supervisor (máximo 3 mínimo 1)
		\UCpaso [\UCactor] Presiona el botón \IUbutton{Guardar}.
		\UCpaso Verifica que ningún campo del formulario este vació \Trayref{C}
		\UCpaso Verifica que el salario sea un dígito.\Trayref{D}
		\UCpaso Verifica que por lo menos este marcada una Sucursal más de las Sucursales marcadas generadas en el paso 9 de este caso de uso.\Trayref{F}
		\UCpaso Muestra la pantalla \IUref{IU13}{Tabla Supervisores}
	\end{UCtrayectoria}

%--------------------------------------		
	\begin{UCtrayectoriaA}{A}{Permiso Denegado}
			\UCpaso Muestra el Mensaje {\bf MSG4-}``Cancelado[{\em Permiso Denegado }].''.
			\UCpaso Muestra la pantalla \IUref{IU13}{Tabla Supervisores}.
		\end{UCtrayectoriaA}
%----------------------------------------
		\begin{UCtrayectoriaA}{B}{No hay ninguna Sucursal Registrada en el sistema }
			\UCpaso Muestra el Mensaje {\bf MSG1-}``Error en la operación [{\em No hay sucursales Registradas aun.}] antes de registrar a algún empleado es necesario que se registre por lo menos una Sucursal, o que exista una Sucursal con puestos vacantes.''.
			\UCpaso[\UCactor] Oprime el botón \IUbutton{ok}.
			\UCpaso Muestra la pantalla \IUref{IU13}{Tabla Supervisores}.
		\end{UCtrayectoriaA}		
%--------------------------------------
		\begin{UCtrayectoriaA}{C}{Existe por lo menos un campo obligatoria que esta vació}
			\UCpaso remarca con Rojo los campos obligatorios que están vacíos 
			y pone la leyenda ``Campo Obligatorio".
			\UCpaso Muestra el Mensaje {\bf MSG1-}``Error en la operación [{\em Campos obligatorios vacíos}] Los campos llenados con Rojo no pueden estar vacíos.''.
			\UCpaso[\UCactor] Oprime el botón \IUbutton{Aceptar}
			\UCpaso Regresa a la pantalla \IUref{IU2}{Formulario Supervisor} Mostrando los campos en rojo y dejando la información introducida en el paso 11 de este caso de uso, como se quedo.
		\end{UCtrayectoriaA}
%--------------------------------------
		\begin{UCtrayectoriaA}{D}{Tipo de Dato Incorrecto}
			\UCpaso remarca con rojo el campo que debe ser dígito, y muestra la leyenda ``Este campo debe ser un dígito" debajo del campo ``salario" en la  pantalla \IUref{IU2}{Formulario Supervisor}.
			\UCpaso Muestra el Mensaje {\bf MSG01-}``Error en la Operación [{\em Tipo de Dato Invalido}]En el campo salario solo se debe introducir números.''.
			\UCpaso Regresa a la pantalla \IUref{IU2}{Formulario Supervisor} Mostrando los campos en rojo y dejando la información introducida en el paso 11 de este caso de uso, como se quedo.
			\UCpaso Continua en el paso 11 de este caso de uso.
		\end{UCtrayectoriaA}
%--------------------------------------
		\begin{UCtrayectoriaA}{F}{Sucursales mal asignadas}
			\UCpaso Crea la leyenda ``Asigna por lo menos una Sucursal" en rojo y la coloca hasta abajo de la pantalla \IUref{IU2}{Formulario Supervisor}
			\UCpaso Muestra el Mensaje {\bf MSG1-}``Error en la Operación [{\em Asignación incorrecta de sucursal}]Verifica que por lo menos se le haya asignado una sucursal al supervisor que se esta registrando.''.
			\UCpaso Regresa a la pantalla \IUref{IU2}{Formulario Supervisor} Mostrando los campos en rojo y, dejando la información introducida en el paso 11 de este caso de uso, como la dejo el [\UCactor].
			\UCpaso Continua en el paso 11 de este caso de uso.
		\end{UCtrayectoriaA}