\begin{UseCase}{CU4}{Desactivar estado de  Empleado}{
		Por razones laborales, es mejor desactivar el estado de un empleado que eliminarlo y revocar el acceso al sistema
	}
		\UCitem{Versión}{\color{Gray}0.1.3}
		\UCitem{Autor}{\color{Gray}Aguilera Rosas Landa Enrique}
		\UCitem{Supervisa}{\color{Gray}Correa Medina Carlos Miguel}
		\UCitem{Actor}{\hyperlink{Alumno}{Dueño}}
		\UCitem{Propósito}{Revocar acceso al sistema del empleado.}
		\UCitem{Entradas}{Nombre del Empleado, Id de Empleado}
		\UCitem{Origen}{Teclado}
		\UCitem{Salidas}{No Aplica.}
		\UCitem{Destino}{Pantalla}
		\UCitem{Precondiciones}{El empleado debe de estar registrado en el sistema y con un estado activo.}
		\UCitem{Postcondiciones}{El empleado perderá su acceso al sistema y su estado cambiara a desactivado .}
		\UCitem{Errores}{La pagina sea inaccesible por el momento debido a fallas con los servidores, Que el empleado tenga su cuenta desactivada}
		\UCitem{Observaciones}{}
		\UCitem{Estado}{En Corrección}
	\end{UseCase}
%--------------------------------------
	\begin{UCtrayectoria}{Principal}
		\UCpaso Incluye el caso de uso \UCref{CU1} paso 11
		\UCpaso[\UCactor] Selecciona La opción de ver Lista de  Empleados presionando el botón \IUbutton{Ver Lista de Empleado}.
		\UCpaso[\UCactor] Introduce el Nombre del empleado a buscar en el campo de Búsqueda y presiona el botón \IUbutton{Buscar} \Trayref{A} .
		\UCpaso Genera y Despliega una lista que coincida con la búsqueda realizada
		\UCpaso[\UCactor] Presiona el botón\IUbutton{Desactivar} del empleado seleccionado.
		\UCpaso Genera y despliega la ventana \IUref{Confirmación Desactivar} para confirmar la operación de desactivar la cuenta seleccionada.
		\UCpaso [\UCactor] Confirma la operación presionando el \IUbutton{Si, Desactivar}
		\UCpaso Redirige al [\UCactor] a la  \IUref{01}{Pantalla Principal de Dueño}.
	\end{UCtrayectoria}


%-------------------------------------------------------------------------


\begin{UCtrayectoriaA}{A}{Empleado no encontrado.}
			\UCpaso Muestra el Mensaje {\bf MSG01-}``Error en la Operación [{\em Empleado no encontrado}] revisa que los campos sean llenados correctamente.''.
			\UCpaso Continúa en el paso 4 del \UCref{CU4}.
		\end{UCtrayectoriaA}
%-------------------------------------------------------------------------

