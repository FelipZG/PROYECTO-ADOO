\begin{UseCase}{CU30}{Modificar Paquete de Descuentos}
{
}
    \UCitem{Versión}
    {
	\color{Gray} 0.1
    }
    \UCitem{Autor}
    {
	\color{Gray}Mart\'inez Quevedo Miguel \'Angel
    }
    \UCitem{Supervisa}
    {
	\color{Gray}
    }
    \UCitem{Actor}
    {
	Due\~no
    }
    \UCitem{Propósito}
    {
	Que el due\~no pueda modificar un paquete de descuentos en caso de error o
	que lo considere necesario.
    }
    \UCitem{Entradas}
    {
	Asociaciones Medicamento-Descuento proporcionadas por el usuario.
    }
    \UCitem{Origen}
    {
	Teclado, Mouse
    }
    \UCitem{Salidas}
    {}
    \UCitem{Destino}
    {}
    \UCitem{Precondiciones}
    {}
    \UCitem{Postcondiciones}
    {
	El paquete de descuentos modificado est\'a registrado en el sistema.
    }
    \UCitem{Errores}
    {}
    \UCitem{Historia de cambio}
    {}
    \UCitem{Observaciones}
    {}
    \UCitem{Estado}
    {
		Aprobado
	}
\end{UseCase}

%Trayectoria Principal

\begin{UCtrayectoria}{Principal}
    \UCpaso Se extiende del caso de uso \UCref{CU0} paso 11
    % el actor selecciona el paquete de descuentos
    \UCpaso Se despliega la pantalla \IUref{IUtodo}{Lista de Paquetes de Descuentos}
    \UCpaso [\UCactor] selecciona un paquete de descuentos del men\'u.
    \UCpaso [\UCactor] presiona el bot\'on \IUbutton{Seleccionar} para confirmar su selecci\'on.
    % el actor modifica el paquete de descuentos
    \UCpaso Se despliega la pantalla \IUref{IUtodo}{Edici\'on de Paquete de Descuentos}
    correspondiente a el paquete seleccionado por el usuario.
    \UCpaso [\UCactor] modifica seg\'un sus necesidades los pares Medicamento-Descuento.
    \UCpaso [\UCactor] presiona el bot\'on \IUbutton{Hecho} para indicar que ha terminado su edici\'on.
    % el actor confirma la modificacion
    \UCpaso Se despliega la pantalla \IUref{IUtodo}{Confirmaci\'on de edici\'on de paquete de
	descuentos} con las ediciones correspondientes.
    \UCpaso [\UCactor] presiona el bot\'on \IUbutton{Aceptar} para indicar que acepta los cambios.
    % el actor es redireccionado a la pantalla principal
    \UCpaso Redirecciona al \UCactor a la  \IUref{IU32}{Pantalla Principal} 
\end{UCtrayectoria}

%Trayectorias Alternativas

%Trayectoria Alternativa C
\begin{UCtrayectoriaA}{}{}
\end{UCtrayectoriaA}

