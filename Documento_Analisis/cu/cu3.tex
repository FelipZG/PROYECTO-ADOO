\begin{UseCase}{CU3}{Registrar cajero}{
		Registramos un nuevo cajero en el sistema, validamos los cajeros por sucursal asi como datos del cajero. 	
	}
		\UCitem{Versión}{\color{Gray}0.1.5}
		\UCitem{Autor}{\color{Gray}Felipe Zamora Gachuz}
		\UCitem{Supervisa}{\color{Gray} }
		\UCitem{Actor}{Dueño}
		\UCitem{Propósito}{Registrar cajeros en la sucursales donde faltan cajeros}
		\UCitem{Entradas}{Correo del cajero,nombre, apellidos del cajero, telefono, salario, sucursal, turno}
		\UCitem{Origen}{Teclado}
		\UCitem{Salidas}{Registro de un cajero en una unica sucursal}
		\UCitem{Destino}{Pantalla}
		\UCitem{Precondiciones}{No tener dado de alta al cajero en el sistema en general, la sucursal debe tener un  numero valido de cajeros en el turno por asignar}
		\UCitem{Postcondiciones}{Registro en el sistema de un cajero nuevo en una sucursal}
		\UCitem{Errores}{No hay respuesta del servidor, no hay conección a internet, el correo ya este registrado en el sistema}
		\UCitem{Observaciones}{}
		\UCitem{Estado}{En revision}
	\end{UseCase}
%--------------------------------------
	\begin{UCtrayectoria}{Principal}
		\UCpaso Incluye el caso de uso \UCref{CU0}{Control de acceso}.
		\UCpaso [\UCactor] Tiene que tener permisos de dueño.
		\UCpaso [\UCactor] Preciona la opcción cajero en el side-bar.
		\UCpaso El sistema genera la lista despliegabre de las sucursales \Trayref{A}.
		\UCpaso El sistema despliega la pantalla \IUref{IU18}{TablaCajeros}, boton  \IUbutton{+Nuevo}.
		\UCpaso El sistema despliega la pantalla \IUref{IU7}{FormularioCajero}.
		\UCpaso [\UCactor] Introduce los datos del formulario.
		\UCpaso [\UCactor] Selecciona la sucursal en la que operara el cajero.
		\UCpaso El sistema genera la lista despliegabre de los  turnos disponibles de la sucursal seleccionada \Trayref{A}.
		\UCpaso [\UCactor] Selecciona el turno del cajero en que se ragistra.
		\UCpaso [\UCactor]Preciona el boton \IUbutton{guardar} \Trayref{H}.
		%---------------
		%---------------
		\UCpaso El sistema verifica el formulario no este vacio \Trayref{B}.
		\UCpaso	El sistema verifica el correo se alfanumerico \Trayref{C}.
		\UCpaso El sistema verifica el correo tenga arroba \Trayref{D}.
		\UCpaso El sistema verifica el correo termine en .com o .mx \Trayref{E}
		\UCpaso El sistema valida que salario sea numerico \Trayref{F}.%numeros en salario		
		\UCpaso EL sistema registra el nuevo cajero en el sistema
		\UCpaso El sistema regresa a la \IUref{IU18}{TablaCajeros}.
		
		
	\end{UCtrayectoria}

%\textbf{NOTA:El numero de cajeros y supervisores es de 3 por sucursal.}
%-------------------------------------------------------------------------

\begin{UCtrayectoriaA}{A}{Sucursal no encontrada}
			\UCpaso El sistema muestra la pantalla {\bf MSG1-``Error en la Operación''}.
			\UCpaso El sistema regresa a la \IUref{IU18}{TablaCajeros}.
		\end{UCtrayectoriaA}
%-------------------------------------------------------------------------

\begin{UCtrayectoriaA}{B}{Hay datos del formulario que estan vacios}
	\UCpaso El sistema muestra abajo en la pantalla el/los nombres de los campos vacios.
	\UCpaso Continua en el paso 7 del \UCref{CU3}.
\end{UCtrayectoriaA}	
%-------------------------------------------------------------------------

\begin{UCtrayectoriaA}{C}{Error no se detecta alfanumericos ates de arroba}
	\UCpaso  El sistema muestra la pantalla {\bf MSG1-``Error en la Operación''}.
	\UCpaso El sistema limpia el campo de correo.
			\UCpaso Continua en el paso 7 del \UCref{CU3}.
		\end{UCtrayectoriaA}
%-------------------------------------------------------------------------

\begin{UCtrayectoriaA}{D}{No se detecto arroba}
	\UCpaso  El sistema muestra la pantalla {\bf MSG1-``Error en la Operación''}.
	\UCpaso El sistema limpia el campo de correo.
		\UCpaso Continua en el paso 7 del \UCref{CU3}.
\end{UCtrayectoriaA}

%-------------------------------------------------------------------------

\begin{UCtrayectoriaA}{E}{No se deteto terminacion .com o .mx}
	\UCpaso  El sistema muestra la pantalla {\bf MSG1-``Error en la Operación''}.
	\UCpaso El sistema limpia el campo de correo.
	 \UCpaso Continua en el paso 7 del \UCref{CU3}.
\end{UCtrayectoriaA}

%-------------------------------------------------------------------------

\begin{UCtrayectoriaA}{F}{Datos alfanumericos detectados}
			\UCpaso  El sistema muestra la pantalla {\bf MSG1-``Error en la Operación''}.
    	\UCpaso El sistema regresa a las pantallas del dueño.
\end{UCtrayectoriaA}

%--------------------------------------
\begin{UCtrayectoriaA}{H}{Preciona el boton cancelar}
	\UCpaso El sistema regresa a la pantalla  \IUref{IU18}{TablaCajeros}.
\end{UCtrayectoriaA}