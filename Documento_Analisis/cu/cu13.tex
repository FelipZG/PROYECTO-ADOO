\begin{UseCase}{CU13}{Consultar Proveedor}{
		La farmacia consta con múltiples proveedores que nos surten medicamentos,  y para poder checar los datos de todos los proveedores que nos venden medicamentos necesitamos un listado de los proveedores con sus datos correspondientes, la consulta del proveedor se realizara mediante una barra de búsqueda la cual estará en la  \IUref{15}{Tabla Proveedores} en la que se introduce ya sea el nombre, Identificador o el correo electrónico del proveedor que deseamos buscar y una vez que presionamos el botón buscar se desplegara una tabla con los datos del proveedor.
	}
		\UCitem{Versión}{\color{Gray}0.8}
		\UCitem{Autor}{\color{Gray}Aguilera Rosas Landa Enrique}
		\UCitem{Supervisa}{\color{Gray} Zamora Gachuz Felipe Jesús}
		\UCitem{Actor}{Dueño}
		\UCitem{Propósito}{Control rápido y eficaz sobre los múltiples proveedores de la Farmacia.}
		\UCitem{Entradas}{Nombre del Proveedor, Correo Electrónico del Proveedor,RFC}
		\UCitem{Origen}{Teclado}
		\UCitem{Salidas}{ Nombre del Proveedor, Correo Electrónico del Proveedor, Teléfono, RFC, Estado.}
		\UCitem{Destino}{Pantalla}
		\UCitem{Precondiciones}{El proveedor debe de estar registrado en el sistema.}
		\UCitem{Postcondiciones}{No Aplica.}
		\UCitem{Errores}{La pagina sea inaccesible por el momento debido a fallas con los servidores, Que el empleado tenga su cuenta no este registrado}
		\UCitem{Observaciones}{}
		\UCitem{Estado}{Revisión}
		\UCitem{Viene de}{CU0}
	\end{UseCase}
%--------------------------------------
	\begin{UCtrayectoria}{Principal}
		\UCpaso Incluye el caso de uso \UCref{CU0}
		\UCpaso[\UCactor] Selecciona La opción compras en la \IUref{01}{Pantalla Principal} presionando el botón \IUbutton{Compras}.
		\UCpaso verifica que los permisos de usuario sean permisos de Dueño. \Trayref{A}
		\UCpaso Despliega las opciones de Compras  \IUref{01}
		\UCpaso [\UCactor] Selecciona La opción Proveedores en la \IUref{01}{Pantalla Principal} presionando el botón \IUbutton{Proveedores}.
		\UCpaso Genera la lista de opciones de búsqueda de proveedor las cuales son: Nombre, RFC y correo
		\UCpaso Genera y Despliega la \IUref{15}{Tabla Proveedores} con el listado de los proveedores con su Nombre , Correo Electrónico , Teléfono, RFC, Estado.
		\UCpaso[\UCactor] Selecciona la opción con la cual buscara al proveedor en la lista desplegable que se genera junto al campo de búsqueda.
		\UCpaso[\UCactor] Introduce el Nombre ó Correo ó RFC del proveedor a buscar en el campo de Búsqueda y Presiona el \IUbutton{Buscar} en la \IUref{15}{Tabla Proveedores}.
		\UCpaso Filtra y Despliega una lista que coincida con la búsqueda realizada en la \IUref{15}{Tabla Proveedores}. según el campo seleccionado \Trayref{B}
		\UCpaso [\UCactor] Presiona la opción Escritorio en la \IUref{15}{Tabla Proveedores}.
		\UCpaso Redirige al [\UCactor] a la  \IUref{01}{Pantalla Principal}.
	\end{UCtrayectoria}

%-------------------------------------------------------------------------
\begin{UCtrayectoriaA}{A}{Permiso denegado.}
			\UCpaso Muestra el Mensaje {\bf MSG04-}``Cancelado [{\em Permiso Denegado}] No cuentas con los permisos necesarios.''.
			\UCpaso Redirige al [\UCactor] a la  \IUref{01}{Pantalla Principal}.
		\end{UCtrayectoriaA}
%-------------------------------------------------------------------------

%-------------------------------------------------------------------------
\begin{UCtrayectoriaA}{B}{Proveedor no encontrado.}
			\UCpaso Muestra el Mensaje {\bf MSG01-}``Error en la Operación [{\em Proveedor no encontrado}]  \IUref{15}{Tabla Proveedores}.''.
			\UCpaso Continúa en el paso 4 del \UCref{CU13}.
		\end{UCtrayectoriaA}
%-------------------------------------------------------------------------
