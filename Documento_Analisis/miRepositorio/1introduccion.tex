\chapter{Introducción}
Este documento contiene la Especificación del proyecto “Fran Farmacias” correspondiente al trabajo realizado en el
semestre 2017-2018-2 para la materia de la Escuela Superior De Computo. Análisis y diseño orientado a objetos en el
grupo 2CV9 por el equipo Lo Que Sea.com

\section{Presentación}
Este proyecto se realiza para la solución de los múltiples problemas que presenta la Franquicia de Farmacias "Fran Farmacias" los cuales son: Problemas de inventario por múltiples perdidas de existencias en el almacén cuya causa es desconocida, Un balance financiero deficiente ya que los saldos de las sucursales tienen variaciones con respecto a las correspondientes, Un control no muy útil sobre los  empleados que operan en las sucursales así como la distribución de medicamentos por parte de los proveedores.

Este documento contiene la especificación de los requerimientos del usuario y del sistema a desarrollar.
Tiene como objetivo establecer la naturaleza y funciones del sistema para su evaluación al final del semestre. 
Este documento debe ser aprobado por los principales responsables del proyecto.
Este documento es el Documento de Análisis del proyecto “Fran Farmacias”.

\section{para qué sirve.}
Este documento sirve para un manejo optimo del sistema, presentamos y describimos la completa operación del sistema asi como sus posibles errores y soluciones.\\
En este documento solo hacemos referencia a la Franquicia de Farmacias “Fran Farmacias”.
\section{Que contiene.}
En este documento se encuentra el análisis completo del sistema, así como los diagramas UML requeridos para el análisis, y las demás tecnologías para este análisis en particular así como políticas propias de la farmacia,
\section{Organización}
Este documento esta dividido por capítulos:\\
2-.Modelo Del Negocio: Presentamos un pequeño contexto del ámbito que representa la Franquicia de Farmacias ``Fran Farmacias'', Los términos del negocio a especificar para evitar problemas con los términos utilizados, El modelo del dominio del problema asi como las reglas presentes del negocio junto con sus procesos.\\
3-.Identificación de requerimientos: Describiremos las necesidades que debe cumplir nuestro sistema así como post y pre-condiciones junto con los comportamientos del sistema.\\
4-.Modelo Dinámico Y Casos de uso: Definimos y especificamos los distintos Actores que tendrán interacción con nuestro sistema así como los distintos y múltiples Casos de uso.\\
5-.Modelo de interacción:Se específican las interfaces de usuario que se ocuparan para realizar el sistema.
mismas que se ocupan para ayudar a especificar formularios y pantallas en la seccion 4, en la parte de casos de uso.
También sirven para fácilitar la explicación del funcionamiento del sistema.\\
6-.Bitácora de Mensajes: Especificamos y definimos los errores o fallas que llega a presentar el sistema; mostramos la pantalla donde se produce tal falla y sus posibles razones.\\ 

\section{Notación, Simbolos y Convenciones Utilizadas}
Los requerimientos funcionales utilizan una clave RFX, donde:\\
X Es un numero consecutivo: 1, 2, 3, ... \\
RF Es la clave para todos los Requerimientos Funcionales.\\\\
Ademas, para los requerimientos funcionales se usan las siguientes abreviaciones.\\
\begin{center}
Id Identificador del requerimiento.\\
Pri. Prioridad\\
Ref. Referencia a los Requerimientos de usuario.\\
MA Prioridad Muy Alta.\\
A Prioridad Alta.\\
M Prioridad Media.\\
B Prioridad Baja.\\
MB Prioridad Muy Baja.\\
\end{center}
Los requerimientos del usuario utilizan una clave RUX, donde:\\
X Es un numero consecutivo: 1, 2, 3, ...\\
RU Es la clave para todos los Requerimientos del Usuario.\\\\
Para los requerimientos del usuario se usan las siguientes abreviaciones.\\
\begin{center}
Los Casos de Uso utilizan una clave CUX, donde:\\
X Es un numero consecutivo: 1, 2, 3, ... \\
CU Es la clave para todos los Casos de Uso.\\
\end{center}
