\begin{UseCase}{CU21}{Cambiar datos de sucursal}{
		Realizar cambios en la información de una sucursal.
	}
		\UCitem{Versión}{\color{Gray}0.1.2}
		\UCitem{Autor}{\color{Gray}Zapata Jasso José Rodolfo}
		\UCitem{Supervisa}{\color{Gray} Correa Medina Carlos Miguel}
		\UCitem{Actor}{\hyperlink{Alumno}{Dueño}}
		\UCitem{Propósito}{Modificar los datos de la sucursal para mantenerla actualizada.}
		\UCitem{Entradas}{Nombre del Empleado, Id de Empleado}
		\UCitem{Origen}{Teclado}
		\UCitem{Salidas}{No Aplica.}
		\UCitem{Destino}{Pantalla}
		\UCitem{Precondiciones}{Debe existir dicha sucursal.}
		\UCitem{Postcondiciones}{Los datos de la sucursal deberan cambiar y en escencia estar actualizados.}
		\UCitem{Errores}{La pagina sera inaccedible por el momento debido a fallas con los servidores.}
		\UCitem{Tipo}{Caso de uso primario}
		\UCitem{Observaciones}{}
		\UCitem{Estado}{aprobado}
	\end{UseCase}


%--------------------------------------


	\begin{UCtrayectoria}{Principal}
		\UCpaso Se incluye del caso de uso \UCref{CU1}.
		\UCpaso Se incluye el caso de uso \UCref{CU22}.
		\UCpaso[\UCactor] Selecciona la opcion Editar, de la sucursal que desea cambair los datos presionando el boton \IUbutton{Editar} que esta representado con un lapiz amarillo.
		\UCpaso Genera y Despliega los datos de la sucursal para su edición. 
		\UCpaso[\UCactor] Realiza los cambios que desea modificar en la sucursal y confirma al presionar el boton\IUbutton{Actualizar} . \Trayref{A}
		\UCpaso Guarda los cambios y redirige al [\UCactor] a la  \IUref{01}{Pantalla Principal de Dueno}.
	\end{UCtrayectoria}


%-------------------------------------------------------------------------


\begin{UCtrayectoriaA}{A}{Error al cambiar el estado.}
			\UCpaso Muestra el Mensaje {\bf MSG1-}`` [{\em Hubo un error en la operación}] ''.
			
		\end{UCtrayectoriaA}
%-------------------------------------------------------------------------

