\begin{UseCase}{CU20}{Desactivar estado de sucursal}{
		Se desactiva cuando la sucursal no estara en operación.
	}
		\UCitem{Versión}{\color{Gray}0.1.2}
		\UCitem{Autor}{\color{Gray}Zapata Jasso José Rodolfo}
		\UCitem{Supervisa}{\color{Gray}Correa Medina Carlos Miguel}
		\UCitem{Actor}{\hyperlink{Alumno}{Dueño}}
		\UCitem{Propósito}{Inhabilitar la sucursal debido a que dejara de de estar en funcionamiento.}
		\UCitem{Entradas}{Nombre del Empleado, Id de Empleado}
		\UCitem{Origen}{Teclado}
		\UCitem{Salidas}{No Aplica.}
		\UCitem{Destino}{Pantalla}
		\UCitem{Precondiciones}{Debe existir dicha sucursal y debe estar activada.}
		\UCitem{Postcondiciones}{El estado de la sucursal deberá cambiar de activada a desactivada.}
		\UCitem{Errores}{La pagina sera inaccesible por el momento debido a fallas con los servidores.}
		\UCitem{Tipo}{Caso de uso primario}
		\UCitem{Observaciones}{}
		\UCitem{Estado}{aprobado}
	\end{UseCase}


%--------------------------------------


	\begin{UCtrayectoria}{Principal}
		\UCpaso Se incluye del caso de uso \UCref{CU1}.
		\UCpaso Se incluye el caso de uso \UCref{CU22}
		\UCpaso[\UCactor] Selecciona la opción Desactivar sucursal, de la sucursal que desea activar presionando el botón \IUbutton{Desactivar} que esta representado con un bote de basura rojo.		
		\UCpaso Genera y despliega una pantalla de confirmación.  {\bf MSG3-}.
		\UCpaso[\UCactor] Confirma la acción seleccionando el botón\IUbutton{Si, Desactivar} . \Trayref{A}
		\UCpaso Guarda los cambios y redirige al [\UCactor] a la  \IUref{01}{Pantalla Principal de Dueño}.
	\end{UCtrayectoria}


%-------------------------------------------------------------------------


\begin{UCtrayectoriaA}{A}{Error al cambiar el estado.}
			\UCpaso Muestra el Mensaje {\bf MSG1-}`` [{\em Hubo un error en la operación}] ''.
			
		\end{UCtrayectoriaA}
%-------------------------------------------------------------------------

