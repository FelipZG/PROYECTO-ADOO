\begin{UseCase}{CU28}{Cerrar turno de caja}{
	Conocer el desempeño de los cajeros a lo largo de su turno.
	}
		\UCitem{Versión}{\color{Gray}0.1.2}
		\UCitem{Autor}{\color{Gray}Correa Medina Carlos Miguel}
		\UCitem{Supervisa}{\color{Gray}}
		\UCitem{Actor}{\hyperlink{Alumno}{Supervisor}}
		\UCitem{Propósito}{Controlar las aperturas de turno de caja en cada cambio de turno de los cajeros}
		\UCitem{Entradas}{fecha  y turno (Matutino,Vespertino,Nocturno),Nombre Completo del Cajero que Finaliza Turno.}
		\UCitem{Origen}{teclado}
		\UCitem{Salidas}{no aplica}
		\UCitem{Destino}{no aplica}
		\UCitem{Precondiciones}{Que anteriormente se aplique el caso de uso \UCref{CU27}}
		\UCitem{Postcondiciones}{El supervisor podrá abrir turno}
		\UCitem{Errores}{La pagina sea inaccesible por el momento debido a fallas con los servidores}
		\UCitem{Tipo}{Caso de uso primario}
		\UCitem{Observaciones}{}
		\UCitem{Estado}{En revisión}
	\end{UseCase}
%--------------------------------------
	\begin{UCtrayectoria}{Principal}
		\UCpaso Se Incluye el caso de uso \UCref{CU1}
		\UCpaso presiona el botón \IUbutton{cerrar turno}
		\UCpaso Verifica que el usuario tenga permisos de supervisor \Trayref{A} 
		\UCpaso Busca y muestra las Sucursales asociadas a la cuenta de supervisor actual.
		\UCpaso [\UCactor] Selecciona la sucursal en la que quiere cerrar turno de caja.
		\UCpaso Cierra el turno de la sucursal y se cancelan las operaciones de cajero sobre esta sucursal ,lo cual implica no dejar que el cajero de el turno correspondiente acceda al sistema. \Trayref{B}
	\end{UCtrayectoria}

%--------------------------------------------------------------
\begin{UCtrayectoriaA}{A}{El actor no tiene permisos de supervisor}
	\UCpaso Muestra el Mensaje {\bf MSG4-}`` [{\em Permiso denegado\textsl{•}}].''.
			\UCpaso Muestra la \IUref{IU1}{Pantalla Principal}
\end{UCtrayectoriaA}
%-----------------------------
\begin{UCtrayectoriaA}{B}{El cajero asignado al turno correspondiente tiene acceso al sistema}
	\UCpaso Muestra el Mensaje {\bf MSG1-}`` [{\em Error de operación\textsl{•}}].''.
			\UCpaso Muestra la \IUref{IU1}{Pantalla Principal}
\end{UCtrayectoriaA}