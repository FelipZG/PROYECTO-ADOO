\begin{UseCase}{CU3}{Eliminar Medicamento}{
		El medicamentos requiere ser eliminado ya que en rara ocasión se solicita
	}
		\UCitem{Versión}{\color{Gray}0.1}
		\UCitem{Autor}{\color{Gray}Zapata Jasso José Rodolfo}
		\UCitem{Supervisa}{\color{Gray}Correa Medina Carlos Miguel}
		\UCitem{Actor}{\hyperlink{Alumno}{Dueño}}
		\UCitem{Propósito}{El dueño eliminara los medicamentos que ya no sean necesarios.}
		\UCitem{Entradas}{Datos del medicamento.}
		\UCitem{Origen}{Teclado}
		\UCitem{Salidas}{Medicamento eliminado.}
		\UCitem{Destino}{Pantalla}
		\UCitem{Precondiciones}{El medicamento debe existir. El actor deberá tener los permisos necesarios para eliminar medicamentos.}
		\UCitem{Postcondiciones}{El medicamento ya no deberá estar registrado.}
		\UCitem{Errores}{}
		\UCitem{Tipo}{Caso de uso primario}
		\UCitem{Observaciones}{}
		\UCitem{Estado}{Aprobado}
	\end{UseCase}
%--------------------------------------
	\begin{UCtrayectoria}{Principal}

		\UCpaso Se extiende del caso de uso \UCref{CU0} paso 11
		\UCpaso Despliega la \IUref{IU34}{Pantalla Principal de dueño} con la lista de Medicamentos Disponibles.
		\UCpaso[\UCactor]Selecciona el medicamento que desea eliminar con el boton\IUbutton{+eliminar}
		\UCpaso Despliega confirmación de eliminar medicamento.\Trayref{A}.
		\UCpaso [\UCactor] Confirma la eliminación de medicamento presionando el botón \IUbutton{Aceptar}.
		\UCpaso Guarda y actualiza la lista de medicamentos.
		\UCpaso Redirecciona al \UCactor a la  \IUref{IU32}{Pantalla Principal} con la lista de Medicamentos Disponibles.
	\end{UCtrayectoria}

%--------------------------------------
		\begin{UCtrayectoriaA}{A}{Imposible eliminar.}
			\UCpaso Muestra el Mensaje {\bf MSG4-}`` [{\em Error al eliminar Medicamento}] revisa que el medicamento se pueda eliminar.''.

		\end{UCtrayectoriaA}


