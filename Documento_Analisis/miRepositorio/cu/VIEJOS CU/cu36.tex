\begin{UseCase}{CU36}{Recuperar Contrase\~na}
{
    Los empleados pueden no recordar su contrase\~na y requieren cambiarla.
}
    \UCitem{Versión}
    {
	\color{Gray} 0.1
    }
    \UCitem{Autor}
    {
	\color{Gray} Miguel \'Angel Mart\'inez Quevedo
    }
    \UCitem{Supervisa}
    {
	\color{Gray}
    }
    \UCitem{Actor}
    {
	Empleado
    }
    \UCitem{Propósito}
    {
	Que el empleado pueda acceder a su cuenta dado que ha olvidado su contrase\~na
    }
    \UCitem{Entradas}
    {
	Nombre de usuario del empleado
    }
    \UCitem{Origen}
    {
	Teclado
    }
    \UCitem{Salidas}
    {}
    \UCitem{Destino}
    {}
    \UCitem{Precondiciones}
    {
	El usuario ha requerido recuperar su contrase\~na
    }
    \UCitem{Postcondiciones}
    {
	El supervisor es notificado de la solicitud de recuperaci\'on de contrase\~na
    }
    \UCitem{Errores}
    {}
    \UCitem{Historia de cambio}
    {}
    \UCitem{Observaciones}
    {}
    \UCitem{Estado}
    {}
\end{UseCase}

%Trayectoria Principal
\begin{UCtrayectoria}{Principal}
    \UCpaso Se extiende del caso de uso \UCref{CU0} paso 11
    \UCpaso Se muestra la pantalla \IUref{IUtodo}{Confirmaci\'on Recuperaci\'on de Contrase\~na} en
    donde el usuario es comunicado que en caso de aceptar, se enviar\'a una notificaci\'on al
    Supervisor.
    \UCpaso El usuario da click en el bot\'on \IUbutton{Aceptar}.
    \UCpaso Se env\'ia una notificaci\'on de la solicitud de recuperaci\'on de contrase\~na al
    Supervisor correspondiente. 
    \UCpaso Se notifica al usuario del \'exito de la notificaci\'on al Supervisor.
    \UCpaso Se redirige al usuario a la pantalla \IUref{IUtodo}{Men\'u Principal}
\end{UCtrayectoria}
% \UCpaso \UCactor \IUref \IUbutton \UCref

%Trayectorias Alternativas
%Trayectoria Alternativa
\begin{UCtrayectoriaA}{A}{El usuario cancel\'o la operaci\'on}
    \UCpaso El usuario presiona el bot\'on IUbutton{Cancelar}.
    \UCpaso Se redirige al usuario a la pantalla \IUref{IUtodo}{Men\'u Principal}
\end{UCtrayectoriaA}

